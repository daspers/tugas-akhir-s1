\chapter{Definisi Masalah Dan Rancangan Algoritma Pencarian Rute Pengantaran}

\section{Definisi Formal Permasalahan}

Diberikan suatu graf berarah dan berbobot $G$ yang memiliki himpunan simpul $V(G)$, himpunan sisi $E(G)$, dan himpunan bobot $W(E(G))$ yang merupakan 
bobot pada tiap sisi di $G$. Diberikan juga $n$ buah pesanan $P$ yang tiap pesanan $pi$ untuk $1 \leq i \leq n$ berupa \textit{tuple} $(a_{i}, b_{i}, c_{i})$, 
dimana $a_{i}$ dan $b_{i}$ merupakan simpul pada graf $G$, $a_{i}$ adalah simpul tempat pengambilan barang pesanan ke-$i$, $b_{i}$ adalah simpul tempat penurunan 
barang pesanan ke-$i$, dan $c_{i}$ adalah ukuran barang yang akan diantar pada pesanan ke-$i$. Terdapat sebuah truk yang dapat digunakan untuk melakukan pengantaran.
Penyimpanan truk memiliki sifat yang sama dengan struktur data \textit{stack}. Truk memiliki kapasitas maksimum penyimpanan 
berupa bilangan bulat positif yang dilambangkan dengan $C$ .

Dari info yang diberikan, masalah yang ingin diselesaikan adalah sebuah jalur yang akan dilewati truk, yang memenuhi syarat berikut:
\begin{enumerate}
  \item Setiap bobot dari sisi pada graf $G$ adalah bilangan bulat positif.
  \item Setiap pesanan memiliki \textit{cost} $c_{i}$ bernilai bilangan bulat positif yang akan memakan kapasitas truk.
  \item Penyimpanan truk bersifat LIFO atau \textit{last in, first out}.
  \item Penurunan barang pada pesanan $p_{i}$ hanya dapat dilakukan setelah barang pada pesanan $p_{i}$ diambil.
  \item Barang yang hanya dapat diturunkan hanya jika berada pada atas penyimpanan.
  \item Jumlah ukuran barang yang terdapat pada penyimpanan suatu truk tidak melebihi kapasitasnya.
\end{enumerate}

Indikator pengukuran dari suatu rute adalah jumlah bobot dari sisi-sisi yang dilalui oleh semua truk yang akan dilambangkan dengan $T$. 
Bobot disini berupa jarak tempuh yang dilalui oleh truk pada rute yang ditentukan
Pencarian rute tersebut memiliki objektif, yaitu meminimalkan nilai dari $T$.

\section{Rancangan Algoritma 1}

Dalam menyelesaikan persoalan tersebut, hal yang perlu diperhatikan hanyalah simpul-simpul \textit{pickup} dan \textit{drop off} pesanan, sehingga graf $G$ 
dapat direduksi menjadi graf $G’$, yang mana simpul-simpul pada graf $G’$ memuat simpul-simpul tempat \textit{pickup} barang dan \textit{drop off} barang.
Sisi-sisi dari graf $G’$ adalah jalur terpendek antara tiap dua simpul pada graf $G'$. 
Perhitungan jalur terpendek ini menggunakan sisi-sisi yang berasal dari graf $G$.

Selanjutnya, hal yang dicari adalah suatu jalur yang memiliki panjang minimal dan memenuhi syarat-syarat dari permasalahan diatas.
Pencarian rute optimal akan dilakukan dengan menggunakan \textit{backtracking} yang dimulai dari setiap simpul \textit{pickup}. 
Algoritma ini menyimpan barang yang berada dalam truk pada struktur data \textit{stack}. Misalkan $S$ adalah simpul-simpul tempat \textit{pickup}.
Dalam melakukan \textit{backtracking}, jika terdapat barang pesanan dalam truk, algoritma ini akan mencoba menelusuri 
kemungkinan jalur ketika melakukan \textit{drop off} barang terakhir pada \textit{stack}. Ketika masih terdapat simpul 
\textit{pickup} pada $S$ dan kapasitas dari truk masih mencukupi,
algoritma ini juga akan mencoba kemungkinan dari setiap simpul pada $S$. Jika mengambil simpul dari $S$, 
maka simpul \textit{drop off}-nya akan ditambahkan ke stack. Terakhir, ketika tidak ada lagi simpul pada $S$ dan
\textit{stack} kosong, maka algoritma ini akan membandingkan mana jalur yang paling baik. Berikut adalah \textit{pseodocode} dari rancangan algoritma 1.

\medskip
\lstinputlisting[style=customc]{codes/ra1.txt}

\section{Rancangan Algoritma 2}
