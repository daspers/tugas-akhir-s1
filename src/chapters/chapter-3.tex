\chapter{Definisi Masalah Dan Rancangan Algoritma Pencarian Rute Optional}

\section{Definisi Formal Permasalahan}

Diberikan suatu graf berarah dan berbobot $G$ yang memiliki himpunan simpul $V(G)$, himpunan sisi $E(G)$, dan himpunan bobot $W(E(G))$ yang merupakan 
bobot pada tiap sisi di $G$. Diberikan juga $n$ buah pesanan $P$ yang tiap pesanan $pi$ untuk $1 \leq i \leq n$ berupa \textit{tuple} $(a_{i}, b_{i}, w_{i})$, 
dimana $a_{i}$ dan $b_{i}$ merupakan simpul pada graf $G$, $a_{i}$ adalah simpul tempat pengambilan barang pesanan ke-$i$, $b_{i}$ adalah simpul tempat penurunan 
barang pesanan ke-$i$, dan $w_{i}$ adalah ukuran barang yang akan diantar pada pesanan ke-$i$. Penyimpanan truk memiliki sifat yang sama dengan struktur data \textit{stack} 
dan mempunyai kapasitas maksimal $M$.

Dari info yang diberikan, masalah yang akan dicari adalah sebuah jalur yang akan dilewati truk, yang memenuhi syarat berikut:
\begin{enumerate}
    \item Penurunan barang pada pesanan $p_{i}$ hanya dapat dilakukan setelah barang pada pesanan $p_{i}$ diambil.
    \item Barang yang hanya dapat diturunkan jika berada pada atas penyimpanan.
    \item Jumlah ukuran barang yang terdapat pada penyimpanan tidak melebihi kapasitas truk, yaitu $M$.
\end{enumerate}

Pencarian rute tersebut memiliki objektif, yaitu jumlah bobot sisi-sisi yang dilalui pada jalur tersebut minimal.

\section{Rancangan Algoritma 1}

Untuk menyelesaikan persoalan tersebut, hal yang perlu diperhatikan adalah simpul-simpul tempat pengambilan dan penurunan barang, sehingga graf $G$ 
dapat direduksi menjadi graf $G’$, yang mana simpul-simpul pada graf $G’$ memuat simpul-simpul tempat pengambilan barang dan penurunan barang serta 
sisi-sisi pada simpul tersebut adalah jalur terpendek antar tiap simpul pada graf $G’$.

Hal pertama yang dilakukan adalah menghitung jalur terpendek, yaitu:
\begin{enumerate}
    \item Jalur terpendek dari simpul tempat pengambilan barang ke simpul tempat pengambilan barang lain.
    \item Jalur terpendek dari simpul tempat pengambilan barang ke simpul tempat penurunan barang.
    \item Jalur terpendek dari simpul tempat penurunan barang ke simpul tempat penurunan barang lain.
\end{enumerate}

Setelah mendapatkan informasi diatas, akan dilakukan pencarian jalur terpendek sehingga semua pesanan 