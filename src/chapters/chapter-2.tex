\chapter{Studi Literatur}

\section{Logistik}

Kegiatan logistik merupakan penyampaian atau pengiriman barang atau material dalam 
jumlah tertentu dan waktu yang tepat ke suatu lokasi tertentu dengan biaya seminimal 
mungkin. Dari proses logistik, material atau barang dapat sampai ke tempat produksi 
melalui saluran distribusi, sehingga mampu memberikan kegunaan yang baik. Logistik yang 
dikelola dengan tepat, baik secara kuantitas maupun kualitas, kuantitas, maupun waktu 
dan biaya, dapan menjadi aset utama organisasi, yaitu sebagai sumber pendapatan yang 
strategis dan berperan mendorong kegiatan ekonomi. Logistik berasal dari kata logis yang 
berarti rasional dan tikos yang berarti berfikir, sehingga logistik memiliki arti 
berfikir rasioal dalam menjalankan kegiatan. 

Menurut Donald Bowersox, Logistik merupakan proses pengelolaan yang strategis terhadap 
pemindahan dan penyimpanan barang dari \textit{suplier} kepada perusahaan dan 
pelanggan. Ciri utama dari kegiatan logistik adalah keterpaduan berbagai dimensi dan 
tuntutan terhadap pemindahan (\textit{movement}) dan penyimpanan (\textit{storage}) yang 
strategis.


\section{\textit{Stack}}

\textit{Stack} adalah suatu struktur data abstrak yang dinamis. Struktur data \textit{stack}
mengikuti aturan \textit{LIFO} (\textit{Last In, First Out}) dalam melakukan operasinya. Terdapat
tiga operasi pada struktur data ini, yaitu \textit{top}, \textit{push} (\textit{insert}), dan
\textit{pop} (\textit{delete}). Operasi \textit{top} mendapatkan \textit{index} atau posisi dari
elemen terakhir yang ditambahkan, \textit{push} menambahkan elemen ke \textit{stack}, dan
\textit{pop} menghapus elemen yang paling terakhir ditambahkan.

Misalkan elemen pada \textit{stack} tersusun atas $S[1, 2, ..., S.top]$, dimana $S[1]$ 
merupakan elemen yang berada pada \textit{bottom} dari \textit{stack} dan $S[S.top]$ 
adalah elemen teratas dari \textit{stack}. \textit{Stack} kosong ketika $S.top$ bernilai 
0. \textit{Pseudocode} dari operasi struktur data \textit{stack} adalah sebagai berikut.

\medskip
\lstinputlisting[style=customc]{codes/stack-empty.txt}
\medskip
\lstinputlisting[style=customc]{codes/stack-push.txt}
\medskip
\lstinputlisting[style=customc]{codes/stack-pop.txt}

\section{\textit{Heap}} 

\textit{Heap} adalah struktur data yang berbentuk pohon ber-akar yang tiap simpul dipohon tersebut
menyimpan suatu nilai. Struktur data ini mendukung operasi-operasi menambahkan elemen atau
\textit{insert}, mendapatkan nilai ekstrim (\textit{find-max}), mengambil dan menghapus nilai
ekstrim dari \textit{heap} (\textit{extract-max}), menghapus nilai extrim (\textit{delete-max}), dan
mengubah nilai extrim dengan nilai baru (\textit{replace}). 

\section{\textit{Priority Queue}}

\textit{Priority queue} adalah struktur data abstrak yang mendukung operasi-operasi
\textit{is-empty}, \textit{push} (\textit{insert} atau \textit{enqueue}), \textit{top}
(\textit{front}), dan \textit{pop}. Setiap elemen yang dimasukkan kedalam \textit{priority queue}
memiliki prioritas, sehingga elemen yang memiliki prioritas tertinggi yang didahulukan. Operasi
\textit{push} memasukkan sebuah elemen beserta dengan prioritasnya. Operasi \textit{top} 

\section{Graf}

Graf adalah struktur diskrit yang terdiri atas simpul dan sisi yang menghubungkan simpul-simpul
tersebut. Graf $G$ dapat didefinisikan sebagai sebuah \textit{tuple} yang terdiri atas himpunan
simpul $V(G)$, himpunan sisi $E(G)$, dan himpunan relasi yang menghubungkan antara tepat dua buah
simpul dan tepat sebuah sisi. Titik ujung dari suatu sisi adalah dua buah simpul yang terhubung pada
sisi tersebut. Cincin adalah sisi yang kedua simpul di ujungnya sama. Graf disebut sederhana jika
tidak memiliki lebih dari satu sisi yang menghubungkan pasangan simpul berbeda yang sama dan tidak
memiliki cincin. Graf berbobot adalah graf yang setiap sisi-sisinya memiliki bobot atau nilai.

Graf berarah adalah graf yang setiap sisi-sisinya dikaitkan dengan suatu arah. Sisi pada graf
berarah dapat di representasikan dengan $e (u, v)$ yang mana sisi $e$ memiliki arah dan
menghubungkan simpul $u$ dan simpul $v$ serta arah dari sisi tersebut adalah dari $u$ ke $v$. Dengan
menggunakan sisi tersebut dari simpul $u$ dapat menuju simpul $v$, namun tidak sebaliknya. Graf yang
setiap sisi-sisinya tidak memiliki arah disebut dengan graf tak berarah.

\section{Jalur}

Jalur dapat didefinisikan sebagai berikut. Misalkan $n$ merupakan bilangan bulat tak negatif dan $G$
merupakan suatu graf. Sebuah jalur dengan panjang $n$ dari simpul $u$ ke simpul $v$ adalah rangkaian
yang tersusun atas $n$ buah sisi $e_{1}, e_{2}, ..., e_{n}$ yang mana untuk setiap $1 \leq i \leq
n$, $e_{i}$ merupakan sisi di $G$ dan sisi $e_{i}$ menghubungkan simpul $v_{i-1}$ dengan simpul
$v_{i}$ serta $x_{0} = u$ dan $x_{n} = v$. Singkatnya, jalur adalah rangkaian dari sisi-sisi yang
dimulai dari suatu simpul dan bergerak dari simpul ke simpul yang terhubung dengan sisi pada graf.
Suatu jalur yang dimulai dan diakhiri pada simpul yang sama disebut dengan sirkuit, yang mana $v_{0}
= v_{n}$ dan minimal melewati sebuah sisi. Suatu jalur disebut sederhana jika sisi yang dilewati
jalur tersebut tidak berulang.

Panjang sebuah jalur pada graf tak berbobot adalah jumlah dari sisi-sisi yang dilewati. 
Panjang dari jalur pada graf berbobot adalah jumlah bobot dari sisi yang dilewati.

\section{Menjelajahi Graf}

Menjelajahi graf adalah suatu proses untuk mengunjungi setiap simpul pada suatu graf. 
Berikut adalah algoritma-algoritma yang sering digunakan untuk menjelajahi graf.

  \subsection{\textit{Breadth-First Search}} Misalkan terdapat sebuah graf $G = (V, E)$ dan sebuah
  simpul awal $s$. Algoritma ini melakukan penjelajahan secara melebar pada setiap \textit{level}.
  Pertama-tama algoritma ini mengunjungi simpul awal $s$ sebagai simpul pada \textit{level} ke-0.
  Lalu, setiap simpul yang yang bersebelahan dengan simpul $s$ akan menjadi simpul-simpul pada
  \textit{level} ke-1 dan algoritma ini akan mengunjungi simpul-simpul tersebut. Setelah itu, semua
  simpul yang bersebelahan dengan simpul-simpul pada \textit{level} ke-1 dan belum pernah
  dikunjungi, akan dijadikan simpul-simpul pada \textit{level} ke-2 dan akan dikunjungi. Hal ini
  dilakukan terus menerus hingga semua simpul yang dapat dikunjungi dari $s$  telah dikunjungi.
  Algoritma ini menggunakan struktur data \textit{queue} dalam melakukan penjelajahan. Berikut
  adalah \textit{pseudocde} dari algoritma \textit{breadth-first search}.

  \medskip
  \lstinputlisting[style=customc, mathescape=true]{codes/bfs.txt}

  \subsection{\textit{Depth-First Search}} Misalkan terdapat sebuah graf $G = (V, E)$, sebuah simpul
  awal $s$, dan $s_{0}, s_{1}, ..., s_{k}$ sebagai simpul-simpul yang bersebelahan dengan $s$.
  Algoritma \textit{depth-first search} menjelajahi graf dengan mengunjungi anak dari suatu simpul
  terlebih dahulu. Pertama-tama algoritma ini mengunjungi simpul $s$. Lalu, memilih salah satu
  simpul, misalkan $s_{0}$, yang bersebelahan dengan $s$ dan belum dikunjungi sebagai simpul awal.
  Setelah tidak ada lagi simpul yang dapat dikujungi dari $s_{0}$, algoritma ini memilih simpul
  lain, misalkan $s_{1}$, yang bersebelahan dengan $s$ dan belum dikunjungi. Hal ini dilakukan
  terus-menerus hingga semua simpul yang dapat dikunjungi dari $s$ telah dikunjungi.

  \medskip
  \lstinputlisting[style=customc, mathescape=true]{codes/dfs.txt}

\section{Persoalan Jalur Terpendek}

Jalur terpendek dari simpul $u$ ke simpul $v$ adalah jalur yang dimulai dari simpul $u$ dan diakhiri
pada simpul $v$ yang memiliki panjang minimum. Persoalan jalur terpendek dapat dibagi menjadi 4
variasi, yaitu persoalan jalur terpendek dari simpul $u$ ke simpul $v$, persoalan jalur terpendek
yang dimulai dari suatu simpul, jalur terpendek untuk setiap pasangan simpul pada graf, dan jalur
terpendek yang berakhir pada suatu simpul. Jalur terpendek memiliki suatu sifat, yaitu sub-jalur
dari suatu jalur terpendek adalah sebuah jalur terpendek. Agar dapat menyimpan jalur terpendek, akan
digunakan $P(V)$ yang mana untuk tiap $p_{i}$ pada $P(V)$ merupakan pendahulu pada jalur terpendek
yang diakhiri di simpul $i$, sehingga untuk mendapatkan jalur terpendek yang diakhiri pada simpul ke
$u$, dapat dilakukan dengan mencari pendahulu-pendahulu dari simpul tersebut.

\medskip
\lstinputlisting[style=customc]{codes/get-path.txt}

Salah satu permasalahan yang dihadapi dalam melakukan pencarian jalur terpendek adalah ketika
terdapat \textit{negative cycle} yang dapat dicapai dari simpul awal. \textit{Negative cycle} adalah
sirkuit yang memiliki panjang yang negatif. Ketika terdapat sirkuit yang dapat dicapai dari $s$,
maka terdapat tak hingga jalur yang dapat dibentuk. Jika sirkuit tersebut memiliki panjang yang
negatif, dapat dibuktikan bahwa tidak ada jalur terpendek yang dimulai dari simpul $s$.

Pada suatu jalur terpendek, mustahil terdapat sirkuit dengan  panjang positif. Misalkan 
$p = [v_{0}, v_{1}, v_{2}, v_{3}, ..., v_{n}]$ adalah jalur terpendek, 
$c = [v_{i}, v_{i+1}, ..., v_{j-1}, v_{j}]$ adalah sirkuit dengan panjang positif yang 
terdapat pada $p$, $L(x)$ adalah panjang dari jalur $x$, dan 
$p’ = [v_{0}, v_{1}, ..., v_{i}, v_{j+1}, …, v_{n}]$ adalah jalur $p$ tanpa sirkuit $c$. 
Jalur $p’$ memiliki panjang $L(p’)$ dan $L(p’) = L(p) - L(c) < L(p)$. Panjang jalur 
$p’$ lebih kecil dari pada panjang jalur $p$, sehingga $p$ tidak mungkin menjadi jalur 
terpendek.

Pada bahasan subbab dibawah ini, akan digunakan teknik relaxation. Teknik ini menyimpan $D(V)$
sebagai batas atas jarak terpendek dari simpul $s$, yang mana di adalah batas atas jarak terpendek
dari simpul $s$ ke simpul $i$. Proses relaxing pada sisi $(u, v)$ terdiri atas mengecek apakah
estimasi jalur terpendek menuju $v$ dapat di perkecil dan mengubah estimasi panjang jalur terpendek.
Selain itu, juga akan dipakai fungsi \textit{init-single-source} yang merupakan fungsi untuk
melakukan inisialisasi pada nilai $P(V)$ dan $D(V)$ serta fungsi relax yang akan menerapkan teknik
\textit{relaxation} pada sisi $(u,v)$.

\medskip
\lstinputlisting[style=customc, mathescape=true]{codes/init-single-source.txt}

\medskip
\lstinputlisting[style=customc]{codes/relax.txt}

Dalam menentukan jalur terpendek maupun panjang jalur terpendek, terdapat beberapa 
algoritma-algoritma seperti berikut.

  \subsection{Algoritma Bellman-Ford}
  Algoritma Bellman-Ford menyelesaikan persoalan jalur terpendek yang dimulai dari suatu simpul
  dengan menggunakan pendekatan pemrograman dinamis. Algoritma Bellman-Ford juga dapat menangani
  kasus \textit{negative cycle}. Diberikan graf $G = (V, E, W)$ dengan simpul awal jalur adalah $s$,
  algoritma Bellman-Ford mengembalikan \textit{boolean true} jika tidak terdapat \textit{negative
  cycle} dalam jalur terpendek dan algoritma ini akan menghasilkan jalur terpendek yang berasal dari
  simpul $s$. Algoritma ini melakukan \textit{}{relaxation} secara terus-menerus terhadap sisi-sisi
  pada graf, sehingga memperkecil nilai estimasi panjang jalur terpendek $D(V)$ hingga mencapai
  panjang jalur terpendek yang sebenarnya. Berikut adalah \textit{pseudocode} dari algoritma
  Bellman-Ford.

  \medskip
  \lstinputlisting[style=customc]{codes/bellman-ford.txt}

  Kompleksitas waktu dari algoritma Bellman-Ford adalah $O(VE)$. Algoritma Bellman-Ford dapat
  digunakan untuk mencari jalur terpendek setiap pasangan simpul pada graf, dengan memanggil fungsi
  Bellman-Ford untuk setiap simpul pada graf.  Dalam hal tersebut kompleksitas waktunya adalah
  $O(VVE)$.

  \subsection{Algoritma Dijkstra}
  Algoritma Dijkstra menyelesaikan persoalan jalur terpendek yang dimulai dari suatu simpul melalui
  pendekatan greedy. Diberikan graf berarah dan berbobot $G = (V, E, W)$, yang mana setiap sisi
  memiliki bobot yang merupakan bilangan tak negatif dan simpul $s$ yang merupakan simpul awal.
  Algoritma Dijkstra menyimpan himpunan simpul $U$ yang mana untuk tiap simpul pada $U$, sudah
  ditentukan jalur terpendek dari $s$. Algoritma ini berulang kali mengambil sebuah simpul $a$ pada
  $V - U$ yang mempunyai jarak terpendek yang telah diestimasi lalu menambahkan $a$ ke himpunan $U$.
  Setelah itu, melakukan relaxation kepada semua sisi yang keluar dari $a$. Berikut adalah
  \textit{pseudocode} dari algoritma Dijkstra.

  \medskip
  \lstinputlisting[style=customc, mathescape=true]{codes/dijkstra.txt}

  Kompleksitas waktu dari algoritma Dijkstra adalah $O(V^{2} + E)$. Ketika menggunakan struktur data
  seperti \textit{self-balancing binary search tree} atau \textit{heap} pada $U$ dan $V$,
  kompleksitas waktu dari algoritma ini dapat mencapai $O((V + E) * \log V)$.

  Algoritma Dijkstra dapat digunakan untuk mencari jalur terpendek setiap pasangan simpul pada graf,
  dengan memanggil fungsi Dijkstra untuk setiap simpul pada graf. Dalam hal tersebut kompleksitas
  waktunya adalah $O(V*(V + E)* \log V)$.

  \subsection{Algoritma Floyd-Warshall}
  Algoritma Floyd-Warshall menggunakan pendekatan pemrograman dinamis dalam menemukan jalur
  terpendek untuk setiap pasangan simpul pada graf berarah dan berbobot $G = (V, E, W)$. Dalam
  menentukan jalur terpendek $(u,v)$, algoritma ini memperhatikan simpul intermediate, yaitu pada
  jalur $p = [v_{1}, v_{2}, v_{3}, ..., v_{n}]$, simpul intermediate adalah simpul yang bukan
  $v_{1}$ ataupun $v_{n}$. Algoritma ini bergantung pada observasi berikut. Misalkan $V = {1, 2, 3,
  ..., n}$ adalah simpul-simpul pada graf, $R = {1, 2, ..., k}$ adalah himpunan bagian dari $V$,
  untuk jalur terpendek dari setiap pasangan $(i, j)$ pada $V$ yang memiliki simpul intermediate
  dari $R$, dan $p$ adalah jalur terpendek dari pasangan-pasangan simpul tersebut. Hubungan dari $k$
  dengan $p$ adalah sebagai berikut.

  \begin{enumerate}
    \item Jika $k$ bukan simpul intermediate dari $p$, maka simpul-simpul intermediate dari $p$ adalah $R’ = {1, 2, ..., k-1}$.
    \item Jika $k$ adalah simpul intermediate dari $p$, jalur terpendek $p$ dapat didekomposisi menjadi jalur terpendek $p^{(1)}$ dari $i$ ke $k$ 
    dan jalur terpendek $p^{(2)}$ dari $k$ ke $j$. Simpul-simpul intermediate dari $p^{(1)}$ dan $p^{(2)}$ adalah dari $R’ = {1, 2, ..., k-1}$.
  \end{enumerate}

  Ketika $k$ bernilai 0, tidak ada simpul \textit{intermediate} pada $p$. Dari observasi di atas,
  dapat dibuat relasi rekurens untuk menghitung jalur terpendek dari $p$, yaitu
  \begin{equation}
    d_{i,j}^{(k)} =
    \begin{cases}
      d_{i,j}^{(0)}, & \text{if } k = 0\\
      min(d_{i,j}^{(k)}, d_{i,k}^{(k-1)} + d_{k,j}^{(k-1)}), & \text{otherwise}.
    \end{cases}
  \end{equation}
  dimana $d_{i,j}^{(n)}$ adalah panjang dari jalur terpendek dari i ke j. Dari relasi rekurens
  diatas, dapat dibuat \textit{pseudocode} dari algoritma Floyd-Warshall sebagai berikut.

  \medskip
  \lstinputlisting[style=customc, mathescape=true]{codes/floyd-warshall.txt}

  Kompleksitas waktu dari algoritma Floyd-Warshall adalah $O(V^{3})$.

\section{\textit{Vehicle Routing Problem}}

\textit{Vehicle Routing Problem} (VRP) merupakan permasalahan optimasi kombinatorial dan pemrograman
integer. Definisi dari permasalahan VRP adalah diberikan satu atau lebih kendaraan dan satu atau
lebih pengangkutan, tentukan rute untuk setiap kendaraan sehingga semua atau sebagian pengangkutan
diselesaikan dengan \textit{cost} seminimal mungkin. Terdapat banyak variasi dari permasalahan VRP,
contoh yang berhubungan dengan topik TA ini adalah sebagai berikut. 

\subsection{\textit{Capacitated Vehicle Routing Problem}}

CVRP merupakan varian VRP yang mana kendaraan yang digunakan memiliki kapasitas. Kapasitas ini bisa
berupa berat maksimal, jumlah barang maksimal, ataupun jumlah pengantaran maksimal.

\subsection{\textit{Vehicle Routing Problem with Pickup And Delivery}}

VRPPD adalah salah satu varian VRP yang tiap permintaan memiliki titik pengambilan
(\textit{pickup}), titik penurunan (\textit{drop off}), dan pengantaran yang perlu dilakukan dari
titik pengambilan ke titik penurunan.